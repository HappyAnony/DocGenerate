%% Generated by Sphinx.
\def\sphinxdocclass{report}
\documentclass[a4paper,10pt,english]{sphinxmanual}
\ifdefined\pdfpxdimen
   \let\sphinxpxdimen\pdfpxdimen\else\newdimen\sphinxpxdimen
\fi \sphinxpxdimen=.75bp\relax

\usepackage[utf8]{inputenc}
\ifdefined\DeclareUnicodeCharacter
 \ifdefined\DeclareUnicodeCharacterAsOptional
  \DeclareUnicodeCharacter{"00A0}{\nobreakspace}
  \DeclareUnicodeCharacter{"2500}{\sphinxunichar{2500}}
  \DeclareUnicodeCharacter{"2502}{\sphinxunichar{2502}}
  \DeclareUnicodeCharacter{"2514}{\sphinxunichar{2514}}
  \DeclareUnicodeCharacter{"251C}{\sphinxunichar{251C}}
  \DeclareUnicodeCharacter{"2572}{\textbackslash}
 \else
  \DeclareUnicodeCharacter{00A0}{\nobreakspace}
  \DeclareUnicodeCharacter{2500}{\sphinxunichar{2500}}
  \DeclareUnicodeCharacter{2502}{\sphinxunichar{2502}}
  \DeclareUnicodeCharacter{2514}{\sphinxunichar{2514}}
  \DeclareUnicodeCharacter{251C}{\sphinxunichar{251C}}
  \DeclareUnicodeCharacter{2572}{\textbackslash}
 \fi
\fi
\usepackage{cmap}
\usepackage[T1]{fontenc}
\usepackage{amsmath,amssymb,amstext}
\usepackage{babel}
\usepackage{times}
\usepackage[Sonny]{fncychap}
\usepackage[dontkeepoldnames]{sphinx}

\usepackage{geometry}

% Include hyperref last.
\usepackage{hyperref}
% Fix anchor placement for figures with captions.
\usepackage{hypcap}% it must be loaded after hyperref.
% Set up styles of URL: it should be placed after hyperref.
\urlstyle{same}

\addto\captionsenglish{\renewcommand{\figurename}{Fig.}}
\addto\captionsenglish{\renewcommand{\tablename}{Table}}
\addto\captionsenglish{\renewcommand{\literalblockname}{Listing}}

\addto\captionsenglish{\renewcommand{\literalblockcontinuedname}{continued from previous page}}
\addto\captionsenglish{\renewcommand{\literalblockcontinuesname}{continues on next page}}

\addto\extrasenglish{\def\pageautorefname{page}}




	\hypersetup{unicode=true}
	\usepackage{CJKutf8}
	\DeclareUnicodeCharacter{00A0}{\nobreakspace}
	\DeclareUnicodeCharacter{2203}{\ensuremath{\exists}}
	\DeclareUnicodeCharacter{2286}{\ensuremath{\subseteq}}
	\DeclareUnicodeCharacter{2713}{x}
	\DeclareUnicodeCharacter{27FA}{\ensuremath{\Longleftrightarrow}}
	\DeclareUnicodeCharacter{221A}{\ensuremath{\sqrt{}}}
	\DeclareUnicodeCharacter{221B}{\ensuremath{\sqrt[3]{}}}
	\DeclareUnicodeCharacter{2295}{\ensuremath{\oplus}}
	\DeclareUnicodeCharacter{2297}{\ensuremath{\otimes}}
	\begin{CJK}{UTF8}{gbsn}
	\AtEndDocument{\end{CJK}}
	

\title{文档编排 Documentation}
\date{2月 10, 2018}
\release{1.0.0}
\author{HappyAnony}
\newcommand{\sphinxlogo}{\vbox{}}
\renewcommand{\releasename}{Release}
\makeindex

\begin{document}

\maketitle
\sphinxtableofcontents
\phantomsection\label{\detokenize{index::doc}}



\chapter{概述}
\label{\detokenize{index:id2}}\label{\detokenize{index:id1}}
\begin{sphinxVerbatim}[commandchars=\\\{\}]
俗话说:不会写文档的程序员都是在耍流氓
\end{sphinxVerbatim}

文档的用途有很多:
\begin{itemize}
\item {} 
知识框架梳理

\item {} 
项目帮助文档

\item {} 
项目API文档

\item {} 
电子书

\end{itemize}

现有的编写文档方式有很多:


\begin{savenotes}\sphinxattablestart
\centering
\begin{tabular}[t]{|\X{15}{60}|\X{15}{60}|\X{30}{60}|}
\hline
\sphinxstylethead{\sphinxstyletheadfamily 
写作方式
\unskip}\relax &\sphinxstylethead{\sphinxstyletheadfamily 
\sphinxhref{http://www.worldhello.net/gotgithub/appendix/markups.html}{标记语言}
\unskip}\relax &\sphinxstylethead{\sphinxstyletheadfamily 
应用场景
\unskip}\relax \\
\hline
\sphinxhref{https://help.github.com/categories/wiki/}{GitHub Wiki}
&
\sphinxhref{https://daringfireball.net/projects/markdown/syntax}{Markdown}
&
知识框架梳理等
\\
\hline
\sphinxhref{https://www.gitbook.com/explore}{Gitbook}
&
\sphinxhref{https://daringfireball.net/projects/markdown/syntax}{Markdown}
&
知识框架梳理、电子书等
\\
\hline
\sphinxhref{http://www.sphinx-doc.org/en/stable/contents.html}{Sphinx}
&
\sphinxhref{http://jwch.sdut.edu.cn/book/rst.html}{reStructuredText} / \sphinxhref{https://daringfireball.net/projects/markdown/syntax}{Markdown}
&
项目帮助文档、项目API文档、电子书等
\\
\hline
\end{tabular}
\par
\sphinxattableend\end{savenotes}

当然,还有其它的方式,比如: \sphinxtitleref{CSDN博客} 、\sphinxtitleref{博客园} 、\sphinxtitleref{私有博客} 等;但是博客有一致命弱点:\sphinxstylestrong{编排杂乱,不便索引}

该系列文章将围绕上述3种主流的文档编排方式进行展开说明。


\chapter{目录}
\label{\detokenize{index:id6}}

\section{sphinx文档编排}
\label{\detokenize{sphinx/index::doc}}\label{\detokenize{sphinx/index:sphinx}}
我自己总结的文档编排分为3个阶段:
\begin{itemize}
\item {} 
文档生成

\item {} 
文档托管

\item {} 
文档发布

\end{itemize}

每个阶段可选的方法和工具有很多,关于使用sphinx做文档编排可参考下列表格:


\begin{savenotes}\sphinxattablestart
\centering
\begin{tabular}[t]{|\X{30}{90}|\X{30}{90}|\X{30}{90}|}
\hline
\sphinxstylethead{\sphinxstyletheadfamily 
文档生成
\unskip}\relax &\sphinxstylethead{\sphinxstyletheadfamily 
文档托管
\unskip}\relax &\sphinxstylethead{\sphinxstyletheadfamily 
文档发布
\unskip}\relax \\
\hline
\sphinxhref{http://www.sphinx-doc.org/en/stable/contents.html}{sphinx}
&
\sphinxhref{https://github.com/}{github} / \sphinxhref{https://about.gitlab.com/}{gitlab} /私有仓库等
&
\sphinxhref{https://readthedocs.org/}{ReadtheDocs} /博客/电子书等
\\
\hline
\end{tabular}
\par
\sphinxattableend\end{savenotes}


\subsection{目录}
\label{\detokenize{sphinx/index:id2}}

\subsubsection{文档生成}
\label{\detokenize{sphinx/1-generate/index::doc}}\label{\detokenize{sphinx/1-generate/index:id1}}

\paragraph{概述}
\label{\detokenize{sphinx/1-generate/index:id2}}\begin{quote}

Sphinx是一个由Python编写的基于 \sphinxhref{http://docutils.sourceforge.net/rst.html}{reStructuredText} 【 \sphinxhref{https://en.wikipedia.org/wiki/Lightweight\_markup\_language}{轻量级标记语言} 的一种 】标记语言的文档生成工具。最早是用来生成Python官方文档,随着工具的完善,它已经成为Python项目首选的文档工具,同时它对C/C++项目也有很好的支持,并计划对其它开发语言添加特殊支持;下面列出了其良好的特性,这些特性在Pytho官方文档中均有体现:
\begin{itemize}
\item {} 
丰富的输出格式: 支持 HTML (包括 Windows 帮助文档), LaTeX (可以打印PDF版本), manual pages(man 文档), 纯文本

\item {} 
完备的交叉引用: 语义化的标签,并可以自动化链接函数,类,引文,术语及相似的片段信息

\item {} 
明晰的分层结构: 可以轻松的定义文档树,并自动化链接同级/父级/下级文章

\item {} 
美观的自动索引: 可自动生成美观的模块索引

\item {} 
精确的语法高亮: 基于 \sphinxhref{http://pygments.org/}{Pygments} 自动生成语法高亮

\item {} 
开放的扩展: 支持代码块的自动测试,并包含Python模块的自述文档(API docs)等

\end{itemize}

Sphinx使用\sphinxhref{http://docutils.sf.net/rst.html}{reStructuredText}作为标记语言, 可以享有\sphinxhref{http://docutils.sf.net/}{Docutils}为\sphinxcode{reStructuredText}提供的分析,转换等多种工具.

Sphinx的具体使用可参考:\sphinxhref{http://www.sphinx-doc.org/en/stable/contents.html}{Sphinx官方文档}以及\sphinxhref{https://zh-sphinx-doc.readthedocs.io/en/latest/contents.html}{Sphinx中文文档}

以下关于Sphinx的使用介绍也是参考上述文档。
\end{quote}


\paragraph{目录}
\label{\detokenize{sphinx/1-generate/index:id6}}

\subparagraph{前言}
\label{\detokenize{sphinx/1-generate/1-introduction::doc}}\label{\detokenize{sphinx/1-generate/1-introduction:id1}}\begin{description}
\item[{参考文档}] \leavevmode\begin{itemize}
\item {} 
\sphinxhref{http://www.sphinx-doc.org/en/stable/contents.html}{Sphinx官方文档}

\item {} 
\sphinxhref{https://www.jianshu.com/p/78e9e1b8553a}{如何用Sphinx快速搭建写书环境}

\item {} 
\sphinxhref{http://www.worldhello.net/gotgithub/appendix/markups.html}{轻量级标记语言}

\item {} 
\sphinxhref{https://zh-sphinx-doc.readthedocs.io/en/latest/contents.html}{Sphinx使用手册}

\end{itemize}

\end{description}


\subparagraph{环境安装}
\label{\detokenize{sphinx/1-generate/2-install::doc}}\label{\detokenize{sphinx/1-generate/2-install:id1}}\begin{description}
\item[{参考文档}] \leavevmode\begin{itemize}
\item {} 
\sphinxhref{http://www.sphinx-doc.org/en/stable/install.html}{Installing Sphinx}

\end{itemize}

\item[{Sphinx是由Python编写的,安装Sphinx之前需要安装下列依赖环境:}] \leavevmode\begin{itemize}
\item {} 
\sphinxhref{https://www.python.org/downloads/}{Python} 环境(Python版本至少在2.7之上)

\item {} 
\sphinxhref{http://docutils.sf.net/}{docutils} 库

\item {} 
\sphinxhref{http://jinja.pocoo.org/}{Jinja2} 库

\item {} 
\sphinxhref{http://pygments.org/}{Pygments} 库(支持源码高亮显示)

\end{itemize}

\end{description}


\subparagraph{Linux安装}
\label{\detokenize{sphinx/1-generate/2-install:linux}}
基本上所有的Linux操作系统都会自带Python环境,如果要安装其它版本的Python环境,可以参考 \sphinxhref{https://docs.python.org/3/using/unix.html}{Python3官方安装文档}

基本上所有的Linux发行版在它们的包仓库都有Sphinx安装包,通常包名叫做 \sphinxcode{python-sphinx} , \sphinxcode{python-Sphinx} 或者 \sphinxcode{sphinx} 。

需要注意的是,这里也有两个带有Sphinx关键字的其它安装包:\sphinxcode{CMU Sphinx} (语音识别工具)、\sphinxcode{Sphinx search} (文本搜索数据库)
\begin{itemize}
\item {} 
Debian/Ubuntu安装Sphinx

\end{itemize}

\begin{sphinxVerbatim}[commandchars=\\\{\}]
\PYGZdl{} apt\PYGZhy{}get install python\PYGZhy{}sphinx
\end{sphinxVerbatim}
\begin{itemize}
\item {} 
RedHat/Centos安装Sphinx

\end{itemize}

\begin{sphinxVerbatim}[commandchars=\\\{\}]
\PYGZdl{} yum install python\PYGZhy{}sphinx
\end{sphinxVerbatim}

当在命令行输入命令 \sphinxcode{sphinx-build -{-}version} , 如果能够回显Sphinx的版本信息,说明Sphinx环境安装成功

\begin{figure}[htbp]
\centering

\noindent\sphinxincludegraphics{{1}.png}
\end{figure}


\subparagraph{Mac安装}
\label{\detokenize{sphinx/1-generate/2-install:mac}}
Mac操作系统都会自带Python环境,如果要安装其它版本的Python环境,可以参考 \sphinxhref{https://docs.python.org/3/using/mac.html}{Python3官方安装文档}
\begin{itemize}
\item {} 
\sphinxhref{https://brew.sh/}{brew} 安装

\end{itemize}

\begin{sphinxVerbatim}[commandchars=\\\{\}]
\PYGZdl{} brew cask install sphinx
\end{sphinxVerbatim}
\begin{itemize}
\item {} 
\sphinxhref{http://www.macports.org/}{MacPorts} 安装

\end{itemize}

\begin{sphinxVerbatim}[commandchars=\\\{\}]
\PYGZdl{} sudo port install py27\PYGZhy{}sphinx
\PYGZdl{} sudo port select \PYGZhy{}\PYGZhy{}set python python27
\PYGZdl{} sudo port select \PYGZhy{}\PYGZhy{}set sphinx py27\PYGZhy{}sphinx
\end{sphinxVerbatim}
\begin{itemize}
\item {} 
\sphinxhref{https://pypi.python.org/pypi/pip}{pip} 安装(pip是python2自带;pip3是python3自带)

\end{itemize}

\begin{sphinxVerbatim}[commandchars=\\\{\}]
\PYGZdl{} pip install sphinx
\PYGZdl{} pip3 install sphinx
\end{sphinxVerbatim}
\begin{itemize}
\item {} 
\sphinxhref{https://www.anaconda.com/download/}{anaconda} 安装(anaconda是一个python开源工具自带了Sphinx环境,所以只需要安装anaconda即可)

\end{itemize}

当在命令行输入命令 \sphinxcode{sphinx-build -{-}version} , 如果能够回显Sphinx的版本信息,说明Sphinx环境安装成功

\begin{figure}[htbp]
\centering

\noindent\sphinxincludegraphics{{1}.png}
\end{figure}


\subparagraph{Windows安装}
\label{\detokenize{sphinx/1-generate/2-install:windows}}\begin{description}
\item[{Windows操作系统没有自带Python环境,可以参考}] \leavevmode\begin{itemize}
\item {} 
\sphinxhref{https://docs.python.org/2/using/windows.html}{Python2官方安装文档}

\item {} 
\sphinxhref{https://docs.python.org/3/using/windows.html}{Python3官方安装文档}

\end{itemize}

\item[{注意:Windows安装Python最好将其安装目录添加到window环境变量PATH中,利于命令行随处执行}] \leavevmode\begin{itemize}
\item {} 
\sphinxhref{https://jingyan.baidu.com/article/cbcede0773e6bb02f50b4d66.html}{添加Python环境变量}

\end{itemize}

\end{description}

python2安装后会自带pip工具;python3安装后会自带pip3工具

接下来使用pip工具安装Sphinx环境

\begin{sphinxVerbatim}[commandchars=\\\{\}]
\PYGZdl{} pip install sphinx
\PYGZdl{} pip3 install sphinx
\end{sphinxVerbatim}
\begin{description}
\item[{也可以使用 \sphinxcode{anaconda} 安装Sphinx环境, \sphinxcode{anaconda} 是一个python开源工具自带了Sphinx环境,所以只需要安装anaconda即可}] \leavevmode\begin{itemize}
\item {} 
\sphinxhref{https://www.anaconda.com/download/}{下载安装}

\end{itemize}

\end{description}

当在命令行输入命令 \sphinxcode{sphinx-build -{-}version} , 如果能够回显Sphinx的版本信息,说明Sphinx环境安装成功

\begin{figure}[htbp]
\centering

\noindent\sphinxincludegraphics{{1}.png}
\end{figure}


\subparagraph{项目创建}
\label{\detokenize{sphinx/1-generate/3-project::doc}}\label{\detokenize{sphinx/1-generate/3-project:id1}}

\subparagraph{参考文档}
\label{\detokenize{sphinx/1-generate/3-project:id2}}\begin{itemize}
\item {} 
\sphinxhref{http://jwch.sdut.edu.cn/book/tool/UseSphinx.html\#id5}{建立sphinx工程}

\end{itemize}


\subparagraph{生成目录}
\label{\detokenize{sphinx/1-generate/3-project:id3}}
安装好Sphinx环境后,它自带了一个工具 \sphinxcode{sphinx-quickstart} , 该工具可以自动生成项目目录文件

在项目目录下执行 \sphinxcode{sphinx-quickstart} 命令

\begin{figure}[htbp]
\centering

\noindent\sphinxincludegraphics{{11}.png}
\end{figure}
\begin{description}
\item[{接下来会进入一个设置向导,根据向导一步一步设置文档项目}] \leavevmode\begin{itemize}
\item {} \begin{description}
\item[{文档根目录(Root path for the documentation),默认为当前目录(.)}] \leavevmode
\begin{figure}[htbp]
\centering

\noindent\sphinxincludegraphics{{2}.png}
\end{figure}

\end{description}

\item {} \begin{description}
\item[{是否分离文档源代码与生成后的文档(Separate source and build directories): y}] \leavevmode
\begin{figure}[htbp]
\centering

\noindent\sphinxincludegraphics{{3}.png}
\end{figure}

\end{description}

\item {} \begin{description}
\item[{模板与静态文件存放目录前缀(Name prefix for templates and static dir):\_}] \leavevmode
\begin{figure}[htbp]
\centering

\noindent\sphinxincludegraphics{{4}.png}
\end{figure}

\end{description}

\item {} \begin{description}
\item[{项目名称(Project name)}] \leavevmode{[}轻量级标记语言;作者名称(Author name):HappyAnony{]}
\begin{figure}[htbp]
\centering

\noindent\sphinxincludegraphics{{5}.png}
\end{figure}

\end{description}

\item {} \begin{description}
\item[{项目版本(Project version)}] \leavevmode{[}1.0.1{]}
\begin{figure}[htbp]
\centering

\noindent\sphinxincludegraphics{{6}.png}
\end{figure}

\end{description}

\item {} \begin{description}
\item[{项目文档语言(Project language):zh\_CN【 \sphinxhref{http://www.sphinx-doc.org/en/stable/config.html\#confval-language}{Sphinx支持语言查看} 】}] \leavevmode
\begin{figure}[htbp]
\centering

\noindent\sphinxincludegraphics{{7}.png}
\end{figure}

\end{description}

\item {} \begin{description}
\item[{文档默认扩展名(Source file suffix)}] \leavevmode{[}.rst{]}
\begin{figure}[htbp]
\centering

\noindent\sphinxincludegraphics{{8}.png}
\end{figure}

\end{description}

\item {} \begin{description}
\item[{默认首页文件名(Name of your master document):index}] \leavevmode
\begin{figure}[htbp]
\centering

\noindent\sphinxincludegraphics{{9}.png}
\end{figure}

\end{description}

\item {} \begin{description}
\item[{是否添加epub目录(Do you want to use the epub builder):y}] \leavevmode
\begin{figure}[htbp]
\centering

\noindent\sphinxincludegraphics{{10}.png}
\end{figure}

\end{description}

\item {} \begin{description}
\item[{启用autodoc\textbar{}doctest\textbar{}intersphinx\textbar{}todo\textbar{}coverage\textbar{}pngmath\textbar{}ifconfig\textbar{}viewcode:y}] \leavevmode
\begin{figure}[htbp]
\centering

\noindent\sphinxincludegraphics{{111}.png}
\end{figure}

\end{description}

\item {} 
生成Makefile (Create Makefile):y

\item {} \begin{description}
\item[{生成windows用命令行(Create Windows command file):y}] \leavevmode
\begin{figure}[htbp]
\centering

\noindent\sphinxincludegraphics{{12}.png}
\end{figure}

\end{description}

\end{itemize}

\end{description}


\subparagraph{目录结构}
\label{\detokenize{sphinx/1-generate/3-project:id5}}
\sphinxcode{sphinx-quickstart} 命令执行完后,项目目录的目录结构如下所示:

\begin{sphinxVerbatim}[commandchars=\\\{\}]
readthedocs
│ make.bat=======make批处理命令
│ Makefile=======make命令执行所依赖的makefile文件
├─build==========运行make命令后,生成的文件都在这个目录里面
└─source=========放置文档的源文件
  │ conf.py====存放生成文档的配置信息
  │ index.rst
  ├─\PYGZus{}static
  └─\PYGZus{}templates
\end{sphinxVerbatim}


\subparagraph{文档编辑}
\label{\detokenize{sphinx/1-generate/4-edit::doc}}\label{\detokenize{sphinx/1-generate/4-edit:id1}}
文档编辑的核心在于 \sphinxcode{文本编辑器} 和 \sphinxcode{reStructuredText语法}


\subparagraph{text编辑器}
\label{\detokenize{sphinx/1-generate/4-edit:text}}\begin{description}
\item[{支持 \sphinxcode{reStructuredText语法} 的文本编辑器有}] \leavevmode\begin{itemize}
\item {} 
\sphinxhref{https://www.sublimetext.com/}{Sublime}

\end{itemize}

\end{description}


\subparagraph{rst基本语法}
\label{\detokenize{sphinx/1-generate/4-edit:rst}}

\subparagraph{参考文档}
\label{\detokenize{sphinx/1-generate/4-edit:id2}}\begin{itemize}
\item {} 
\sphinxhref{http://www.sphinx-doc.org/en/stable/contents.html}{官方文档}

\item {} 
\sphinxhref{http://www.worldhello.net/gotgithub/appendix/markups.html}{轻量级标记语言}

\end{itemize}


\subparagraph{标题}
\label{\detokenize{sphinx/1-generate/4-edit:id5}}\begin{itemize}
\item {} 
一级标题:\sphinxcode{========}

\item {} 
二级标题:\sphinxcode{-{-}-{-}-{-}-{-}}

\item {} 
三级标题:\sphinxcode{\textasciitilde{}\textasciitilde{}\textasciitilde{}\textasciitilde{}\textasciitilde{}\textasciitilde{}\textasciitilde{}\textasciitilde{}}

\item {} 
四级标题:\sphinxcode{\textasciicircum{}\textasciicircum{}\textasciicircum{}\textasciicircum{}\textasciicircum{}\textasciicircum{}\textasciicircum{}\textasciicircum{}}

\item {} 
五级标题:\sphinxcode{++++++++}

\item {} 
六级标题:\sphinxcode{{}`{}`{}`{}`{}`{}`{}`{}`}

\end{itemize}


\subparagraph{段落}
\label{\detokenize{sphinx/1-generate/4-edit:id6}}\begin{itemize}
\item {} 
空行分段

\end{itemize}


\begin{savenotes}\sphinxattablestart
\centering
\begin{tabular}[t]{|\X{30}{60}|\X{30}{60}|}
\hline
\sphinxstylethead{\sphinxstyletheadfamily 
代码示例
\unskip}\relax &\sphinxstylethead{\sphinxstyletheadfamily 
输出示例
\unskip}\relax \\
\hline\begin{sphinxfigure-in-table}
\centering

\noindent\sphinxincludegraphics{{13}.png}
\end{sphinxfigure-in-table}\relax
&\begin{sphinxfigure-in-table}
\centering

\noindent\sphinxincludegraphics{{21}.png}
\end{sphinxfigure-in-table}\relax
\\
\hline
\end{tabular}
\par
\sphinxattableend\end{savenotes}
\begin{itemize}
\item {} 
回车自动续行

\end{itemize}


\begin{savenotes}\sphinxattablestart
\centering
\begin{tabular}[t]{|\X{30}{60}|\X{30}{60}|}
\hline
\sphinxstylethead{\sphinxstyletheadfamily 
代码示例
\unskip}\relax &\sphinxstylethead{\sphinxstyletheadfamily 
输出示例
\unskip}\relax \\
\hline\begin{sphinxfigure-in-table}
\centering

\noindent\sphinxincludegraphics{{31}.png}
\end{sphinxfigure-in-table}\relax
&\begin{sphinxfigure-in-table}
\centering

\noindent\sphinxincludegraphics{{41}.png}
\end{sphinxfigure-in-table}\relax
\\
\hline
\end{tabular}
\par
\sphinxattableend\end{savenotes}
\begin{itemize}
\item {} 
不留白续行

\end{itemize}


\begin{savenotes}\sphinxattablestart
\centering
\begin{tabular}[t]{|\X{30}{60}|\X{30}{60}|}
\hline
\sphinxstylethead{\sphinxstyletheadfamily 
代码示例
\unskip}\relax &\sphinxstylethead{\sphinxstyletheadfamily 
输出示例
\unskip}\relax \\
\hline\begin{sphinxfigure-in-table}
\centering

\noindent\sphinxincludegraphics{{51}.png}
\end{sphinxfigure-in-table}\relax
&\begin{sphinxfigure-in-table}
\centering

\noindent\sphinxincludegraphics{{61}.png}
\end{sphinxfigure-in-table}\relax
\\
\hline
\end{tabular}
\par
\sphinxattableend\end{savenotes}
\begin{itemize}
\item {} 
插入换行

\end{itemize}


\begin{savenotes}\sphinxattablestart
\centering
\begin{tabular}[t]{|\X{30}{60}|\X{30}{60}|}
\hline
\sphinxstylethead{\sphinxstyletheadfamily 
代码示例
\unskip}\relax &\sphinxstylethead{\sphinxstyletheadfamily 
输出示例
\unskip}\relax \\
\hline\begin{sphinxfigure-in-table}
\centering

\noindent\sphinxincludegraphics{{71}.png}
\end{sphinxfigure-in-table}\relax
&\begin{sphinxfigure-in-table}
\centering

\noindent\sphinxincludegraphics{{81}.png}
\end{sphinxfigure-in-table}\relax
\\
\hline\begin{sphinxfigure-in-table}
\centering

\noindent\sphinxincludegraphics{{91}.png}
\end{sphinxfigure-in-table}\relax
&\begin{sphinxfigure-in-table}
\centering

\noindent\sphinxincludegraphics{{101}.png}
\end{sphinxfigure-in-table}\relax
\\
\hline
\end{tabular}
\par
\sphinxattableend\end{savenotes}
\begin{itemize}
\item {} 
段落缩进

\end{itemize}


\begin{savenotes}\sphinxattablestart
\centering
\begin{tabular}[t]{|\X{30}{60}|\X{30}{60}|}
\hline
\sphinxstylethead{\sphinxstyletheadfamily 
代码示例
\unskip}\relax &\sphinxstylethead{\sphinxstyletheadfamily 
输出示例
\unskip}\relax \\
\hline\begin{sphinxfigure-in-table}
\centering

\noindent\sphinxincludegraphics{{112}.png}
\end{sphinxfigure-in-table}\relax
&\begin{sphinxfigure-in-table}
\centering

\noindent\sphinxincludegraphics{{121}.png}
\end{sphinxfigure-in-table}\relax
\\
\hline\begin{sphinxfigure-in-table}
\centering

\noindent\sphinxincludegraphics{{131}.png}
\end{sphinxfigure-in-table}\relax
&\begin{sphinxfigure-in-table}
\centering

\noindent\sphinxincludegraphics{{14}.png}
\end{sphinxfigure-in-table}\relax
\\
\hline
\end{tabular}
\par
\sphinxattableend\end{savenotes}


\subparagraph{代码块}
\label{\detokenize{sphinx/1-generate/4-edit:id7}}

\begin{savenotes}\sphinxattablestart
\centering
\begin{tabular}[t]{|\X{30}{60}|\X{30}{60}|}
\hline
\sphinxstylethead{\sphinxstyletheadfamily 
代码示例
\unskip}\relax &\sphinxstylethead{\sphinxstyletheadfamily 
输出示例
\unskip}\relax \\
\hline\begin{sphinxfigure-in-table}
\centering

\noindent\sphinxincludegraphics{{15}.png}
\end{sphinxfigure-in-table}\relax
&\begin{sphinxfigure-in-table}
\centering

\noindent\sphinxincludegraphics{{16}.png}
\end{sphinxfigure-in-table}\relax
\\
\hline
\end{tabular}
\par
\sphinxattableend\end{savenotes}


\subparagraph{列表}
\label{\detokenize{sphinx/1-generate/4-edit:id8}}\begin{itemize}
\item {} 
无序列表

\end{itemize}


\begin{savenotes}\sphinxattablestart
\centering
\begin{tabular}[t]{|\X{30}{60}|\X{30}{60}|}
\hline
\sphinxstylethead{\sphinxstyletheadfamily 
代码示例
\unskip}\relax &\sphinxstylethead{\sphinxstyletheadfamily 
输出示例
\unskip}\relax \\
\hline\begin{sphinxfigure-in-table}
\centering

\noindent\sphinxincludegraphics{{17}.png}
\end{sphinxfigure-in-table}\relax
&\begin{sphinxfigure-in-table}
\centering

\noindent\sphinxincludegraphics{{18}.png}
\end{sphinxfigure-in-table}\relax
\\
\hline
\end{tabular}
\par
\sphinxattableend\end{savenotes}
\begin{itemize}
\item {} 
有序列表

\end{itemize}


\begin{savenotes}\sphinxattablestart
\centering
\begin{tabular}[t]{|\X{30}{60}|\X{30}{60}|}
\hline
\sphinxstylethead{\sphinxstyletheadfamily 
代码示例
\unskip}\relax &\sphinxstylethead{\sphinxstyletheadfamily 
输出示例
\unskip}\relax \\
\hline\begin{sphinxfigure-in-table}
\centering

\noindent\sphinxincludegraphics{{19}.png}
\end{sphinxfigure-in-table}\relax
&\begin{sphinxfigure-in-table}
\centering

\noindent\sphinxincludegraphics{{20}.png}
\end{sphinxfigure-in-table}\relax
\\
\hline
\end{tabular}
\par
\sphinxattableend\end{savenotes}
\begin{itemize}
\item {} 
列表续行、段落和代码块

\end{itemize}


\begin{savenotes}\sphinxattablestart
\centering
\begin{tabular}[t]{|\X{30}{60}|\X{30}{60}|}
\hline
\sphinxstylethead{\sphinxstyletheadfamily 
代码示例
\unskip}\relax &\sphinxstylethead{\sphinxstyletheadfamily 
输出示例
\unskip}\relax \\
\hline\begin{sphinxfigure-in-table}
\centering

\noindent\sphinxincludegraphics{{211}.png}
\end{sphinxfigure-in-table}\relax
&\begin{sphinxfigure-in-table}
\centering

\noindent\sphinxincludegraphics{{22}.png}
\end{sphinxfigure-in-table}\relax
\\
\hline
\end{tabular}
\par
\sphinxattableend\end{savenotes}


\subparagraph{定义}
\label{\detokenize{sphinx/1-generate/4-edit:id9}}

\begin{savenotes}\sphinxattablestart
\centering
\begin{tabular}[t]{|\X{30}{60}|\X{30}{60}|}
\hline
\sphinxstylethead{\sphinxstyletheadfamily 
代码示例
\unskip}\relax &\sphinxstylethead{\sphinxstyletheadfamily 
输出示例
\unskip}\relax \\
\hline\begin{sphinxfigure-in-table}
\centering

\noindent\sphinxincludegraphics{{23}.png}
\end{sphinxfigure-in-table}\relax
&\begin{sphinxfigure-in-table}
\centering

\noindent\sphinxincludegraphics{{24}.png}
\end{sphinxfigure-in-table}\relax
\\
\hline
\end{tabular}
\par
\sphinxattableend\end{savenotes}


\subparagraph{字体}
\label{\detokenize{sphinx/1-generate/4-edit:id10}}\begin{itemize}
\item {} 
粗体和斜体

\end{itemize}


\begin{savenotes}\sphinxattablestart
\centering
\begin{tabular}[t]{|\X{30}{60}|\X{30}{60}|}
\hline
\sphinxstylethead{\sphinxstyletheadfamily 
代码示例
\unskip}\relax &\sphinxstylethead{\sphinxstyletheadfamily 
输出示例
\unskip}\relax \\
\hline\begin{sphinxfigure-in-table}
\centering

\noindent\sphinxincludegraphics{{25}.png}
\end{sphinxfigure-in-table}\relax
&\begin{sphinxfigure-in-table}
\centering

\noindent\sphinxincludegraphics{{26}.png}
\end{sphinxfigure-in-table}\relax
\\
\hline
\end{tabular}
\par
\sphinxattableend\end{savenotes}
\begin{itemize}
\item {} 
删除线

\end{itemize}


\begin{savenotes}\sphinxattablestart
\centering
\begin{tabular}[t]{|\X{30}{60}|\X{30}{60}|}
\hline
\sphinxstylethead{\sphinxstyletheadfamily 
代码示例
\unskip}\relax &\sphinxstylethead{\sphinxstyletheadfamily 
输出示例
\unskip}\relax \\
\hline\begin{sphinxfigure-in-table}
\centering

\noindent\sphinxincludegraphics{{27}.png}
\end{sphinxfigure-in-table}\relax
&\begin{sphinxfigure-in-table}
\centering

\noindent\sphinxincludegraphics{{28}.png}
\end{sphinxfigure-in-table}\relax
\\
\hline
\end{tabular}
\par
\sphinxattableend\end{savenotes}
\begin{itemize}
\item {} 
下划线

\end{itemize}


\begin{savenotes}\sphinxattablestart
\centering
\begin{tabular}[t]{|\X{30}{60}|\X{30}{60}|}
\hline
\sphinxstylethead{\sphinxstyletheadfamily 
代码示例
\unskip}\relax &\sphinxstylethead{\sphinxstyletheadfamily 
输出示例
\unskip}\relax \\
\hline\begin{sphinxfigure-in-table}
\centering

\noindent\sphinxincludegraphics{{29}.png}
\end{sphinxfigure-in-table}\relax
&\begin{sphinxfigure-in-table}
\centering

\noindent\sphinxincludegraphics{{30}.png}
\end{sphinxfigure-in-table}\relax
\\
\hline
\end{tabular}
\par
\sphinxattableend\end{savenotes}
\begin{itemize}
\item {} 
上标、下标

\end{itemize}


\begin{savenotes}\sphinxattablestart
\centering
\begin{tabular}[t]{|\X{30}{60}|\X{30}{60}|}
\hline
\sphinxstylethead{\sphinxstyletheadfamily 
代码示例
\unskip}\relax &\sphinxstylethead{\sphinxstyletheadfamily 
输出示例
\unskip}\relax \\
\hline\begin{sphinxfigure-in-table}
\centering

\noindent\sphinxincludegraphics{{311}.png}
\end{sphinxfigure-in-table}\relax
&\begin{sphinxfigure-in-table}
\centering

\noindent\sphinxincludegraphics{{32}.png}
\end{sphinxfigure-in-table}\relax
\\
\hline
\end{tabular}
\par
\sphinxattableend\end{savenotes}
\begin{itemize}
\item {} 
等宽字体

\end{itemize}


\begin{savenotes}\sphinxattablestart
\centering
\begin{tabular}[t]{|\X{30}{60}|\X{30}{60}|}
\hline
\sphinxstylethead{\sphinxstyletheadfamily 
代码示例
\unskip}\relax &\sphinxstylethead{\sphinxstyletheadfamily 
输出示例
\unskip}\relax \\
\hline\begin{sphinxfigure-in-table}
\centering

\noindent\sphinxincludegraphics{{33}.png}
\end{sphinxfigure-in-table}\relax
&\begin{sphinxfigure-in-table}
\centering

\noindent\sphinxincludegraphics{{34}.png}
\end{sphinxfigure-in-table}\relax
\\
\hline
\end{tabular}
\par
\sphinxattableend\end{savenotes}
\begin{itemize}
\item {} 
引言

\end{itemize}


\begin{savenotes}\sphinxattablestart
\centering
\begin{tabular}[t]{|\X{30}{60}|\X{30}{60}|}
\hline
\sphinxstylethead{\sphinxstyletheadfamily 
代码示例
\unskip}\relax &\sphinxstylethead{\sphinxstyletheadfamily 
输出示例
\unskip}\relax \\
\hline\begin{sphinxfigure-in-table}
\centering

\noindent\sphinxincludegraphics{{35}.png}
\end{sphinxfigure-in-table}\relax
&\begin{sphinxfigure-in-table}
\centering

\noindent\sphinxincludegraphics{{36}.png}
\end{sphinxfigure-in-table}\relax
\\
\hline
\end{tabular}
\par
\sphinxattableend\end{savenotes}
\begin{itemize}
\item {} 
清除标记空白

\end{itemize}


\begin{savenotes}\sphinxattablestart
\centering
\begin{tabular}[t]{|\X{30}{60}|\X{30}{60}|}
\hline
\sphinxstylethead{\sphinxstyletheadfamily 
代码示例
\unskip}\relax &\sphinxstylethead{\sphinxstyletheadfamily 
输出示例
\unskip}\relax \\
\hline\begin{sphinxfigure-in-table}
\centering

\noindent\sphinxincludegraphics{{37}.png}
\end{sphinxfigure-in-table}\relax
&\begin{sphinxfigure-in-table}
\centering

\noindent\sphinxincludegraphics{{38}.png}
\end{sphinxfigure-in-table}\relax
\\
\hline
\end{tabular}
\par
\sphinxattableend\end{savenotes}


\subparagraph{链接}
\label{\detokenize{sphinx/1-generate/4-edit:id11}}\begin{itemize}
\item {} 
URL自动链接

\end{itemize}


\begin{savenotes}\sphinxattablestart
\centering
\begin{tabular}[t]{|\X{30}{60}|\X{30}{60}|}
\hline
\sphinxstylethead{\sphinxstyletheadfamily 
代码示例
\unskip}\relax &\sphinxstylethead{\sphinxstyletheadfamily 
输出示例
\unskip}\relax \\
\hline\begin{sphinxfigure-in-table}
\centering

\noindent\sphinxincludegraphics{{39}.png}
\end{sphinxfigure-in-table}\relax
&\begin{sphinxfigure-in-table}
\centering

\noindent\sphinxincludegraphics{{40}.png}
\end{sphinxfigure-in-table}\relax
\\
\hline
\end{tabular}
\par
\sphinxattableend\end{savenotes}
\begin{itemize}
\item {} 
文字链接

\end{itemize}


\begin{savenotes}\sphinxattablestart
\centering
\begin{tabular}[t]{|\X{30}{60}|\X{30}{60}|}
\hline
\sphinxstylethead{\sphinxstyletheadfamily 
代码示例
\unskip}\relax &\sphinxstylethead{\sphinxstyletheadfamily 
输出示例
\unskip}\relax \\
\hline\begin{sphinxfigure-in-table}
\centering

\noindent\sphinxincludegraphics{{411}.png}
\end{sphinxfigure-in-table}\relax
&\begin{sphinxfigure-in-table}
\centering

\noindent\sphinxincludegraphics{{42}.png}
\end{sphinxfigure-in-table}\relax
\\
\hline
\end{tabular}
\par
\sphinxattableend\end{savenotes}
\begin{itemize}
\item {} 
内部跳转

\end{itemize}


\begin{savenotes}\sphinxattablestart
\centering
\begin{tabular}[t]{|\X{30}{60}|\X{30}{60}|}
\hline
\sphinxstylethead{\sphinxstyletheadfamily 
代码示例
\unskip}\relax &\sphinxstylethead{\sphinxstyletheadfamily 
输出示例
\unskip}\relax \\
\hline\begin{sphinxfigure-in-table}
\centering

\noindent\sphinxincludegraphics{{43}.png}
\end{sphinxfigure-in-table}\relax
&\begin{sphinxfigure-in-table}
\centering

\noindent\sphinxincludegraphics{{44}.png}
\end{sphinxfigure-in-table}\relax
\\
\hline
\end{tabular}
\par
\sphinxattableend\end{savenotes}
\begin{itemize}
\item {} 
脚注

\end{itemize}


\begin{savenotes}\sphinxattablestart
\centering
\begin{tabular}[t]{|\X{30}{60}|\X{30}{60}|}
\hline
\sphinxstylethead{\sphinxstyletheadfamily 
代码示例
\unskip}\relax &\sphinxstylethead{\sphinxstyletheadfamily 
输出示例
\unskip}\relax \\
\hline\begin{sphinxfigure-in-table}
\centering

\noindent\sphinxincludegraphics{{45}.png}
\end{sphinxfigure-in-table}\relax
&\begin{sphinxfigure-in-table}
\centering

\noindent\sphinxincludegraphics{{46}.png}
\end{sphinxfigure-in-table}\relax
\\
\hline
\end{tabular}
\par
\sphinxattableend\end{savenotes}


\subparagraph{图片}
\label{\detokenize{sphinx/1-generate/4-edit:id12}}

\begin{savenotes}\sphinxattablestart
\centering
\begin{tabular}[t]{|\X{30}{60}|\X{30}{60}|}
\hline
\sphinxstylethead{\sphinxstyletheadfamily 
代码示例
\unskip}\relax &\sphinxstylethead{\sphinxstyletheadfamily 
输出示例
\unskip}\relax \\
\hline\begin{sphinxfigure-in-table}
\centering

\noindent\sphinxincludegraphics{{47}.png}
\end{sphinxfigure-in-table}\relax
&\begin{sphinxfigure-in-table}
\centering

\noindent\sphinxincludegraphics{{48}.png}
\end{sphinxfigure-in-table}\relax
\\
\hline
\end{tabular}
\par
\sphinxattableend\end{savenotes}


\subparagraph{表格}
\label{\detokenize{sphinx/1-generate/4-edit:id13}}

\begin{savenotes}\sphinxattablestart
\centering
\begin{tabular}[t]{|\X{30}{60}|\X{30}{60}|}
\hline
\sphinxstylethead{\sphinxstyletheadfamily 
代码示例
\unskip}\relax &\sphinxstylethead{\sphinxstyletheadfamily 
输出示例
\unskip}\relax \\
\hline\begin{sphinxfigure-in-table}
\centering

\noindent\sphinxincludegraphics{{49}.png}
\end{sphinxfigure-in-table}\relax
&\begin{sphinxfigure-in-table}
\centering

\noindent\sphinxincludegraphics{{50}.png}
\end{sphinxfigure-in-table}\relax
\\
\hline
\end{tabular}
\par
\sphinxattableend\end{savenotes}


\subparagraph{其它}
\label{\detokenize{sphinx/1-generate/4-edit:id14}}\begin{itemize}
\item {} 
转义

\end{itemize}


\begin{savenotes}\sphinxattablestart
\centering
\begin{tabular}[t]{|\X{30}{60}|\X{30}{60}|}
\hline
\sphinxstylethead{\sphinxstyletheadfamily 
代码示例
\unskip}\relax &\sphinxstylethead{\sphinxstyletheadfamily 
输出示例
\unskip}\relax \\
\hline\begin{sphinxfigure-in-table}
\centering

\noindent\sphinxincludegraphics{{511}.png}
\end{sphinxfigure-in-table}\relax
&\begin{sphinxfigure-in-table}
\centering

\noindent\sphinxincludegraphics{{52}.png}
\end{sphinxfigure-in-table}\relax
\\
\hline
\end{tabular}
\par
\sphinxattableend\end{savenotes}


\subparagraph{注意}
\label{\detokenize{sphinx/1-generate/4-edit:id15}}\begin{itemize}
\item {} 
各种标记字符与其它文本内容之间最好留一个空格,否则会报语法警告

\end{itemize}


\subparagraph{生成文档}
\label{\detokenize{sphinx/1-generate/5-build::doc}}\label{\detokenize{sphinx/1-generate/5-build:id1}}

\subparagraph{conf配置}
\label{\detokenize{sphinx/1-generate/5-build:conf}}

\subparagraph{生成html}
\label{\detokenize{sphinx/1-generate/5-build:html}}

\subparagraph{reStructuredText扩展}
\label{\detokenize{sphinx/1-generate/6-markup:restructuredtext}}\label{\detokenize{sphinx/1-generate/6-markup::doc}}

\subsubsection{文档托管}
\label{\detokenize{sphinx/2-collocation/index::doc}}\label{\detokenize{sphinx/2-collocation/index:id1}}

\paragraph{目录}
\label{\detokenize{sphinx/2-collocation/index:id2}}

\subparagraph{github托管}
\label{\detokenize{sphinx/2-collocation/1-github/index::doc}}\label{\detokenize{sphinx/2-collocation/1-github/index:github}}

\subparagraph{目录}
\label{\detokenize{sphinx/2-collocation/1-github/index:id1}}

\subparagraph{本地仓库搭建}
\label{\detokenize{sphinx/2-collocation/1-github/1-InitLocal::doc}}\label{\detokenize{sphinx/2-collocation/1-github/1-InitLocal:id1}}

\subparagraph{参考文档}
\label{\detokenize{sphinx/2-collocation/1-github/1-InitLocal:id2}}\begin{itemize}
\item {} 
\sphinxhref{https://git-scm.com/book/zh/v2}{官方说明文档}

\item {} 
\sphinxhref{http://pan.baidu.com/s/1bpzQBV5}{相关pdf文档}

\item {} 
\sphinxhref{http://www.liaoxuefeng.com/wiki/0013739516305929606dd18361248578c67b8067c8c017b000/}{廖雪峰博客}

\end{itemize}


\subparagraph{git下载安装}
\label{\detokenize{sphinx/2-collocation/1-github/1-InitLocal:git}}\begin{itemize}
\item {} \begin{description}
\item[{window下载安装}] \leavevmode\begin{itemize}
\item {} 
\sphinxhref{https://git-scm.com/download/win}{官网下载} {[}国内直接从官网下载比较困难,有时需要翻墙{]}

\item {} 
\sphinxhref{https://github.com/waylau/git-for-win}{github仓库下载}

\end{itemize}

\end{description}

\item {} \begin{description}
\item[{linux下载安装}] \leavevmode\begin{itemize}
\item {} 
\sphinxhref{https://git-scm.com/download/linux}{Download for Linux and Unix}

\end{itemize}

\end{description}

\item {} \begin{description}
\item[{Mac下载安装}] \leavevmode\begin{itemize}
\item {} 
\sphinxhref{https://git-scm.com/download/mac}{官网下载}

\end{itemize}

\end{description}

\end{itemize}


\subparagraph{初始化仓库}
\label{\detokenize{sphinx/2-collocation/1-github/1-InitLocal:id7}}
window下命令行工具也有很多,比较推荐的是
\begin{itemize}
\item {} 
Git Bash{[}安装git后自带的bash环境,支持shell命令{]}

\item {} 
\sphinxhref{http://cmder.net/}{Cmder} {[}window上的命令行神器{]}

\end{itemize}

在这里我们使用 \sphinxcode{git bash}

切换到文档主目录{[}Makefile文件所在目录{]},点击鼠标右键,选择Git Bash Here,使用git自带bash环境,输入 \sphinxcode{git init} 命令初始化仓库

\begin{figure}[htbp]
\centering

\noindent\sphinxincludegraphics{{110}.png}
\end{figure}

\begin{figure}[htbp]
\centering

\noindent\sphinxincludegraphics{{210}.png}
\end{figure}

这时Git就把仓库建好了,而且告诉你是一个空的仓库 \sphinxcode{empty Git repository} ,可以发现当前目录下多了一个 \sphinxcode{.git} 的文件夹,这个目录是Git来跟踪管理版本库的,没事千万不要手动修改这个目录里面的文件,否则就把Git仓库给破坏了。如果你没有看到 \sphinxcode{.git} 文件夹,那是因为这个目录默认是隐藏的,用 \sphinxcode{ls -ah} 命令就可以看见隐藏的 \sphinxcode{.git} 文件夹

\begin{figure}[htbp]
\centering

\noindent\sphinxincludegraphics{{310}.png}
\end{figure}


\subparagraph{提交本地仓库}
\label{\detokenize{sphinx/2-collocation/1-github/1-InitLocal:id8}}\begin{itemize}
\item {} 
初始化仓库后,我们需要将目录下所有文件进行版本控制,输入 \sphinxcode{git add -{-}all} 将当前目录下所有文件加入到版本控制中;之后可以使用 \sphinxcode{git status} 命令查看当前版本控制状态,确认是否将所有文件都添加到版本控制当中

\end{itemize}

\begin{figure}[htbp]
\centering

\noindent\sphinxincludegraphics{{410}.png}
\end{figure}
\begin{itemize}
\item {} 
确认所有文件都添加到版本控制中后,使用 \sphinxcode{git commit -am "\textless{}message\textgreater{}"} 命令将当前所有版本控制的文件{[}即上图红框标注的待commit的文件{]}commit提交到本地git仓库中

\end{itemize}

\begin{figure}[htbp]
\centering

\noindent\sphinxincludegraphics{{53}.png}
\end{figure}
\begin{itemize}
\item {} 
再使用 \sphinxcode{git status} 命令查看当前版本控制状态,确认没有待commit的文件,working工作区完整clean

\end{itemize}

\begin{figure}[htbp]
\centering

\noindent\sphinxincludegraphics{{62}.png}
\end{figure}


\subparagraph{配置远程仓库}
\label{\detokenize{sphinx/2-collocation/1-github/2-ConfigRemote::doc}}\label{\detokenize{sphinx/2-collocation/1-github/2-ConfigRemote:id1}}
参考文档
\begin{itemize}
\item {} 
\sphinxhref{https://jingyan.baidu.com/article/948f592414ad67d80ef5f966.html}{github多帐号配置SSH key}

\end{itemize}


\subparagraph{注册github账号}
\label{\detokenize{sphinx/2-collocation/1-github/2-ConfigRemote:github}}
登录 \sphinxhref{https://github.com/}{github官网}

如果你已经有账号,那么点击右上角的 \sphinxcode{sign in} 直接登录

如果没有账号,依次输入昵称、邮箱、密码,然后点击 \sphinxcode{Sign up for GitHub} 进行注册,按照默认的设置完成注册,最后还要进行邮件确认,我们登录到自己的注册邮箱中,会有一个github发来的邮件,点击即可

\begin{figure}[htbp]
\centering

\noindent\sphinxincludegraphics{{72}.png}
\end{figure}


\subparagraph{建立远程仓库}
\label{\detokenize{sphinx/2-collocation/1-github/2-ConfigRemote:id3}}
使用刚刚注册的github账号登录,点击 \sphinxcode{Start a project} ,在 \sphinxcode{Repository name} 下输入仓库名,其他的保持默认即可,然后点击 \sphinxcode{Create repository} ,到这里,你在github上的的一个仓库就已经建立成功

\begin{figure}[htbp]
\centering

\noindent\sphinxincludegraphics{{82}.png}
\end{figure}

\begin{figure}[htbp]
\centering

\noindent\sphinxincludegraphics{{92}.png}
\end{figure}


\subparagraph{配置git用户名和邮箱}
\label{\detokenize{sphinx/2-collocation/1-github/2-ConfigRemote:git}}
使用以下命令将申请的github账号用户名、邮箱和git进行绑定
\begin{itemize}
\item {} 
\sphinxcode{git config -{-}global user.name "Your\_User\_Name"}

\item {} 
\sphinxcode{git config -{-}global user.email "Your\_Email"}

\end{itemize}

\begin{figure}[htbp]
\centering

\noindent\sphinxincludegraphics{{102}.png}
\end{figure}

\begin{figure}[htbp]
\centering

\noindent\sphinxincludegraphics{{113}.png}
\end{figure}


\subparagraph{配置SSH Key}
\label{\detokenize{sphinx/2-collocation/1-github/2-ConfigRemote:ssh-key}}
git和github连接的方式有两种
\begin{itemize}
\item {} 
http连接:默认情况下,当我们\sphinxcode{git push}将本地仓库同步到远程仓库时,git将采用http方式来连接github,但是此时需要输入用户名和密码验证git提交的合法性,这样会很不方便

\item {} 
ssh连接:使用\sphinxcode{ssh-keygen}工具生成一对密钥,私钥保存本地,公钥配置到github仓库,\sphinxcode{git push}将本地仓库同步到远程仓库时,无需输入用户名和密码,通过私钥公钥验证即可验证合法性

\end{itemize}

关于ssh相关命令行工具的使用可参考{\color{red}\bfseries{}{}`ssh详解 \textless{}\textgreater{}{}`\_}

这里我们将选择ssh方式连接github仓库,此时就需要配置\sphinxcode{SSH Key};此处可以参考\sphinxhref{https://help.github.com/articles/connecting-to-github-with-ssh/}{Github官方文档}配置\sphinxcode{SSH Key}

配置\sphinxcode{SSH Key}的一般思路是
\begin{itemize}
\item {} 
使用\sphinxcode{ssh-keygen}工具生成一对秘钥(公钥和私钥),默认存放在用户家目录的\sphinxcode{.ssh}目录下

\item {} 
将生成的\sphinxcode{.pub}公钥文件打开,将文件内容添加到github仓库中

\item {} 
使用\sphinxcode{ssh -T}工具验证配置是否生效

\end{itemize}

在实际应用中,我们可以将仓库配置类型分类两大类
\begin{itemize}
\item {} 
单用户多仓库配置

\item {} 
多用户单仓库配置

\end{itemize}

如果我们一个用户只使用自己的一个github仓库,那就好办,只需要生成一对密钥即可;但是实际应用中,我们要么是维护自己账户的多个仓库,要么是需要多个用户维护同一个仓库,此时就需要管理多对密钥。那么该如何管理呢?
\begin{itemize}
\item {} 
生成密钥:\sphinxcode{ssh-keygen}工具生成密钥是默认是将私钥保存到\sphinxcode{id\_rsa}文件中,将公钥保存到\sphinxcode{id\_rsa.pub}文件中;所以在生成其它github仓库或者其它github用户密钥时,指定其它文件名,不要使用默认值,否则前一个密钥会被后一个密钥覆盖

\item {} 
密钥添加:将生成的私钥使用\sphinxcode{ssh-agent}工具添加到\sphinxcode{ssh agent}引擎中,因为ssh只能识别使用\sphinxcode{ssh agent}引擎中的私钥,默认只添加了文件名为\sphinxcode{id\_rsa}的私钥

\item {} 
配置config文件:config文件用来配置指定仓库的相关信息,是由多个Host字段组成,以供\sphinxcode{git push}将本地仓库同步到远程仓库时遍历使用,config文件Host字段格式如下:

\end{itemize}

\begin{sphinxVerbatim}[commandchars=\\\{\}]
\PYGZsh{} DocGenerate仓库(15377649725@163.com)配置======注释信息
Host DocGenerate =================================定义Host字段别名,在配置remote url时用来替换github.com
        HostName github.com=======================指定托管平台域名
        PreferredAuthentications publickey========指定仓库认证方式为公钥认证
        IdentityFile \PYGZti{}/.ssh/id\PYGZus{}rsa\PYGZus{}docgenerate====指定私钥文件路径
\end{sphinxVerbatim}

ssh秘钥验证逻辑是
\begin{enumerate}
\item {} 
查找\sphinxcode{.ssh}目录下是否存在\sphinxcode{config} 文件,如果存在执行第2步;如果不存在则执行第3步

\item {} 
从上到下依次遍历读取\sphinxcode{config}文件内容,找到当前\sphinxcode{git remote -v}命令配置的仓库对应的\sphinxcode{Host字段},在该字段中找到对应的私钥文件,判断该私钥文件是否在\sphinxcode{ssh-agent}中,如果在执行第4步,否则报错

\item {} 
判断当前目录是否存储\sphinxcode{id\_rsa}文件,如果存在执行第4步,否则报错

\item {} 
判断当前读取的私钥文件是否和github仓库的公钥匹配上,如果匹配,则验证通过;否则验证失败

\end{enumerate}


\subparagraph{单用户多仓库配置}
\label{\detokenize{sphinx/2-collocation/1-github/2-ConfigRemote:id5}}\begin{itemize}
\item {} 
使用\sphinxcode{ssh-keygen -t rsa -C "Github的注册邮箱地址"}命令生成\sphinxcode{DocGenerate}仓库对应的密钥对,指定生成路径

\end{itemize}

\begin{figure}[htbp]
\centering

\noindent\sphinxincludegraphics{{122}.png}
\end{figure}
\begin{itemize}
\item {} 
使用\sphinxcode{ssh-keygen -t rsa -C "Github的注册邮箱地址"}命令生成\sphinxcode{GnuToolchain}仓库对应的密钥对,指定生成路径

\end{itemize}

\begin{figure}[htbp]
\centering

\noindent\sphinxincludegraphics{{132}.png}
\end{figure}
\begin{itemize}
\item {} 
使用\sphinxcode{ssh-add Path\_To\_Private\_Key}命令将刚生成的各个私钥文件添加到\sphinxcode{ssh agent}引擎中

\end{itemize}

\begin{figure}[htbp]
\centering

\noindent\sphinxincludegraphics{{141}.png}
\end{figure}
\begin{itemize}
\item {} 
如果出现\sphinxcode{Could not open a connection to your authentication agent}的错误,就执行\sphinxcode{ssh-agent bash}命令;然后再执行上述命令

\end{itemize}

\begin{figure}[htbp]
\centering

\noindent\sphinxincludegraphics{{151}.png}
\end{figure}
\begin{itemize}
\item {} 
在\sphinxcode{\textasciitilde{}/.ssh}目录下找到config文件,如果没有就使用\sphinxcode{touch config}命令创建该文件,编辑如下

\end{itemize}

\begin{figure}[htbp]
\centering

\noindent\sphinxincludegraphics{{161}.png}
\end{figure}
\begin{itemize}
\item {} \begin{description}
\item[{进入\sphinxcode{.ssh}文件夹,使用文本编辑器(sublime/notepad++)打开\sphinxcode{id\_rsa\_docgenerate.pub}文件和\sphinxcode{id\_rsa\_GnuToolchain.pub}文件,复制全部内容(即生成的公钥)}] \leavevmode\begin{itemize}
\item {} 
千万不要使用Windows自带的记事本编辑任何文本文件。原因是Microsoft开发记事本的团队使用了一个非常弱智的行为来保存\sphinxcode{UTF-8}编码的文件,他们自作聪明地在每个文件开头添加了\sphinxcode{0xefbbbf}(十六进制)的字符

\end{itemize}

\end{description}

\end{itemize}

\begin{figure}[htbp]
\centering

\noindent\sphinxincludegraphics{{171}.png}
\end{figure}

\begin{figure}[htbp]
\centering

\noindent\sphinxincludegraphics{{181}.png}
\end{figure}
\begin{itemize}
\item {} 
登录github账号,然后在页面右下角的\sphinxcode{Your repositories}目录下,打开创建的仓库,点击右上角的\sphinxcode{settings},左边选中\sphinxcode{Deploy keys},右边点击\sphinxcode{Add deploy key},\sphinxcode{Title}可随便填写,\sphinxcode{Key}粘贴上面复制的key(即.pub文件的全部内容),然后点击\sphinxcode{Add key}

\end{itemize}

\begin{figure}[htbp]
\centering

\noindent\sphinxincludegraphics{{191}.png}
\end{figure}

\begin{figure}[htbp]
\centering

\noindent\sphinxincludegraphics{{201}.png}
\end{figure}
\begin{itemize}
\item {} 
之后注册邮箱会收到一封github官网发来的验证邮件,打开邮件链接进行确认即可

\item {} 
使用\sphinxcode{ssh -T git@DocGenerate}命令和\sphinxcode{ssh -T git@GnuToolchain}命令验证配置是否成功

\end{itemize}

\begin{figure}[htbp]
\centering

\noindent\sphinxincludegraphics{{212}.png}
\end{figure}
\begin{itemize}
\item {} 
如果是第一次,会提示是否\sphinxcode{continue},输入yes就会看到:\sphinxcode{You’ve successfully username, but GitHub does not provide shell access} (username会显示为你的账号名称)这就表示已成功连上github

\end{itemize}


\subparagraph{多用户单仓库配置}
\label{\detokenize{sphinx/2-collocation/1-github/2-ConfigRemote:id6}}

\subparagraph{远程仓库部署}
\label{\detokenize{sphinx/2-collocation/1-github/3-DeployRemote::doc}}\label{\detokenize{sphinx/2-collocation/1-github/3-DeployRemote:id1}}
参考文档
\begin{itemize}
\item {} 
\sphinxhref{https://www.jianshu.com/p/835e0a48c825}{如何解决failed to push some refs to git}

\item {} 
\sphinxhref{http://blog.csdn.net/huahua78/article/details/52330792}{fatal Could not read from remote repository的解决办法}

\end{itemize}


\subparagraph{配置remote url}
\label{\detokenize{sphinx/2-collocation/1-github/3-DeployRemote:remote-url}}\begin{itemize}
\item {} 
可使用\sphinxcode{git remote  -v}命令查看当前配置的remote仓库

\item {} 
点击github仓库页面的\sphinxcode{clone and download};然后点击\sphinxcode{Use SSH},找到ssh的clone地址,复制该地址

\end{itemize}

\begin{figure}[htbp]
\centering

\noindent\sphinxincludegraphics{{231}.png}
\end{figure}
\begin{itemize}
\item {} \begin{description}
\item[{如果\sphinxcode{git remote  -v}命令结果为空,则使用\sphinxcode{git remote add origin \textless{}url\textgreater{}}命令将刚才复制的\sphinxcode{remote ssh url}新添加到git中}] \leavevmode\begin{itemize}
\item {} 
\sphinxcode{origin}为remote远程仓库的默认名称

\item {} 
注意url中的\sphinxcode{github.com}需替换成config文件中Host别名\sphinxcode{DocGenerate}或\sphinxcode{GnuToolchain}

\end{itemize}

\end{description}

\end{itemize}

\begin{figure}[htbp]
\centering

\noindent\sphinxincludegraphics{{241}.png}
\end{figure}
\begin{itemize}
\item {} \begin{description}
\item[{如果\sphinxcode{git remote  -v}命令结果不为空,则使用\sphinxcode{git remote set-url origin \textless{}url\textgreater{}}命令将\sphinxcode{remote url}重置成刚才复制的\sphinxcode{remote ssh url}}] \leavevmode\begin{itemize}
\item {} 
注意url中的\sphinxcode{github.com}需替换成config文件中Host别名\sphinxcode{DocGenerate}或\sphinxcode{GnuToolchain}

\end{itemize}

\end{description}

\end{itemize}

\begin{figure}[htbp]
\centering

\noindent\sphinxincludegraphics{{251}.png}
\end{figure}
\begin{itemize}
\item {} 
使用\sphinxcode{git push -u origin master}命令将本地仓库push同步到刚才配置的remote远程仓库中

\end{itemize}

\begin{figure}[htbp]
\centering

\noindent\sphinxincludegraphics{{261}.png}
\end{figure}
\begin{itemize}
\item {} \begin{description}
\item[{此时可能会报错误\sphinxcode{fatal: Could not read from remote repository.},出现错误的主要原因是github中的\sphinxcode{README.md}文件不在本地代码目录中,可使用\sphinxcode{git pull -{-}rebase origin master}命令解决}] \leavevmode\begin{itemize}
\item {} 
此时需要确保当前working暂存区的所有文件都已经commit提交到本地仓库中

\item {} 
该项也提醒我们以后凡是\sphinxcode{git push}之前最后先使用\sphinxcode{git pull}将remote远程仓库内容拉到本地和本地仓库合并,然后再push到远程仓库,避免push时出现错误

\end{itemize}

\end{description}

\end{itemize}

\begin{figure}[htbp]
\centering

\noindent\sphinxincludegraphics{{271}.png}
\end{figure}
\begin{itemize}
\item {} 
然后使用\sphinxcode{git push -u origin master}命令再次将本地仓库push同步到刚才配置的remote远程仓库中,此时就会push成功

\end{itemize}

\begin{figure}[htbp]
\centering

\noindent\sphinxincludegraphics{{281}.png}
\end{figure}


\subparagraph{gitlab托管}
\label{\detokenize{sphinx/2-collocation/2-gitlab/index:gitlab}}\label{\detokenize{sphinx/2-collocation/2-gitlab/index::doc}}

\subparagraph{本地仓库托管}
\label{\detokenize{sphinx/2-collocation/3-local/index::doc}}\label{\detokenize{sphinx/2-collocation/3-local/index:id1}}

\subsubsection{文档发布}
\label{\detokenize{sphinx/3-release/index::doc}}\label{\detokenize{sphinx/3-release/index:id1}}

\paragraph{目录}
\label{\detokenize{sphinx/3-release/index:id2}}

\subparagraph{ReadtheDocs}
\label{\detokenize{sphinx/3-release/1-docs:readthedocs}}\label{\detokenize{sphinx/3-release/1-docs::doc}}

\subparagraph{电子书}
\label{\detokenize{sphinx/3-release/2-ebook::doc}}\label{\detokenize{sphinx/3-release/2-ebook:id1}}

\section{Gitbook文档编排}
\label{\detokenize{gitbook/index:gitbook}}\label{\detokenize{gitbook/index::doc}}
和sphinx文档编排一样,Gitbook文档编排也分为3个阶段:
\begin{itemize}
\item {} 
文档生成

\item {} 
文档托管

\item {} 
文档发布

\end{itemize}


\subsection{目录}
\label{\detokenize{gitbook/index:id1}}

\subsubsection{生成文档}
\label{\detokenize{gitbook/1-generate/index::doc}}\label{\detokenize{gitbook/1-generate/index:id1}}

\subsubsection{文档托管}
\label{\detokenize{gitbook/2-collocation/index::doc}}\label{\detokenize{gitbook/2-collocation/index:id1}}

\subsubsection{文档发布}
\label{\detokenize{gitbook/3-release/index::doc}}\label{\detokenize{gitbook/3-release/index:id1}}
……………………\sphinxstyleemphasis{持续更新中}………………….



\renewcommand{\indexname}{Index}
\printindex
\end{document}